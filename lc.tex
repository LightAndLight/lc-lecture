\documentclass[aspectratio=169]{beamer}
\usepackage{amsmath, tikz, graphicx, cancel}

\newcommand{\bluebox}[1]{\tikz[baseline] \node[fill=blue!20,rectangle,anchor=text]{#1}; }
\newcommand{\boxon}[3]{ \temporal<#2>{\whitebox{$#3$}}{#1{$#3$}}{\whitebox{$#3$}} }
\newcommand{\whitebox}[1]{\tikz[baseline] \node[fill=\bgcolor,rectangle,anchor=text]{#1}; }
\newcommand{\blueboxon}[2]{ \boxon{\bluebox}{#1}{#2} }

\newcommand{\bgcolor}{black!3}
\setbeamercolor{background canvas}{bg=\bgcolor}

\addtobeamertemplate{frametitle}{\let\insertframetitle\insertsectionhead}{}

\defbeamertemplate{section page}{minimal}[1][]{
  \begin{centering}{}
    \vskip1em\par
    \begin{beamercolorbox}[sep=12pt,center]{part title}
      \usebeamerfont{section title}\insertsection\par
    \end{beamercolorbox}
  \end{centering}
}

\setbeamertemplate{section page}[minimal]
\AtBeginSection{\frame{\sectionpage}}

\title{Lambda Calculus}
\author{Isaac Elliott}

\begin{document}
\beamertemplatenavigationsymbolsempty

% should hand-poll their familiarity with lambda calculus,
% from 'never heard of it' to 'my code is just one giant lambda'
\frame{\titlepage}

\section{What is computation?}

\begin{frame}
  \framesubtitle{Turing machine}
  \frametitle

  \begin{itemize}
    \item Infinite tape of initially blank cells
    \item Program state
    \item Current focus
    \item Given the current focus and the program state:
    \begin{enumerate}
      \item Write to the current focus
      \item Shift focus to one of its neighbours
      \item Halt, or continue from a specified program state
    \end{enumerate}
  \end{itemize}

\end{frame}

\begin{frame}
  \framesubtitle{$\lambda$-calculus}
  \frametitle

  \only<1-4>{
  Three expression forms
  \begin{enumerate}
    \item<2-4> $ \lambda x. \; e $ (abstraction)
    \item<3-4> $ a \; b $ (application)
    \item<4> $ x $ (variable)
  \end{enumerate}
  }

  \only<5-9>{
  A syntactic operation (substitution)
  \begin{itemize}
    \item<6-9> $ x[x\backslash a] \leadsto a $
    \item<7-9> $ y[x\backslash a] \leadsto y \;\;\; (\text{given} \; y \neq x) $
    \item<8-9> $ (m \; n)[x\backslash a] \leadsto m[x\backslash a] \; n[x\backslash a]$
    \item<9> $ (\lambda y. \; e)[x\backslash a] \leadsto \lambda y. \; e[x\backslash a] \;\;\; (\text{given} \; y \neq x) $
  \end{itemize}
  }

  \only<10-11>{
  A reduction rule (beta reduction)
  \begin{itemize}
    \item<11> $ (\lambda x. e) \; a \Rightarrow e[x\backslash a] $ 
  \end{itemize}
  }

\end{frame}

\begin{frame}
  \framesubtitle{$\lambda$-calculus}
  \frametitle

  \only<1>{
  \[ f(x) = \dots \]
  \[ f = \lambda x. \; \dots \]
  }

  \only<2>{
  \[ g(x, y) = \dots \]
  \[ g = \lambda x. \; \lambda y. \; \dots \]
  }

\end{frame}

\begin{frame}
  \frametitle

  % Do you think there are some Turing machine algorithms that couldn't be
  % expressed? Or maybe vice-versa?
  %
  % Some Turing machine programs never halt. Can we create a lambda calculus
  % expression that never stops reducing?

  How does $\lambda$-calculus compare to Turing machines in terms of computability?

  \begin{itemize}
    % I won't prove it, but I will show you some tools that might make this
    % claim seem plausible
    \item<2-> They are equivalent (Church-Turing thesis)
  \end{itemize}

\end{frame}

\section{Anatomy of $\lambda$-calculus}

\begin{frame}
  \framesubtitle{Abstraction}
  \frametitle

  { \Huge
    \[ \blueboxon{2}{\lambda} \blueboxon{3}{x} \blueboxon{2}{.} \blueboxon{4}{e} \]
  }

\end{frame}

\begin{frame}
  \framesubtitle{Application}
  \frametitle

  { \Huge
    \[ \blueboxon{1}{a} \blueboxon{2}{b} \]
  }

\end{frame}

\begin{frame}
  \framesubtitle{Variable}
  \frametitle

  { \Huge \[ x \] }
\end{frame}

\begin{frame}
  \framesubtitle{Examples - terms}
  \frametitle

  \begin{itemize}
    \item $ \lambda x. \; x $
    \item $ \lambda y. \; y \; y $
    \item $ (\lambda a. \; a) \; (\lambda b. \; b) $
    \item $ \lambda f. \; \lambda x. \; f \; x $
  \end{itemize}
\end{frame}

\begin{frame}
  \framesubtitle{Examples - beta reduction}
  \frametitle

  \[ \lambda x. \; x \]
\end{frame}

\begin{frame}
  \framesubtitle{Examples - beta reduction}
  \frametitle

  \[ (\lambda x. \; x) \; (\lambda y. \; y) \]
  \[ x[x\backslash \lambda y. \; y] \leadsto \lambda y. \; y \]
\end{frame}

\begin{frame}
  \framesubtitle{Examples - beta reduction}
  \frametitle

  \[ (\lambda a. \; a \; a) \; (\lambda b. \; b) \]
  \[
  (a \; a)[a\backslash \lambda b. \; b] \leadsto
  a[a\backslash \lambda b. \; b] \; a[a\backslash \lambda b. \; b] \leadsto
  (\lambda b. \; b) \; (\lambda b. \; b)
  \]
  \[ (\lambda b. \; b) \; (\lambda c. \; c) \]
  \[ b[b\backslash\lambda c. \; c] \leadsto \lambda c. \; c \]
\end{frame}

\begin{frame}
  \framesubtitle{Examples - beta reduction}
  \frametitle

  \[ (\lambda x. \; \lambda y. \; x) \; (\lambda z. \; z) \]
  \[
  (\lambda y. \; x)[x\backslash \lambda z. \; z] \leadsto
  \lambda y. \; x[x\backslash \lambda z. \; z] \leadsto
  \lambda y. \; \lambda z. \; z
  \]
\end{frame}

\section{Church Encoding}

\begin{frame}
  \framesubtitle{Booleans}
  \frametitle

  % Booleans values come in two flavours: true and false
  %
  % It doesn't matter what we call them, so long as we have two different values
  % that support common operations (negation, conjunction, disjunction, etc.)

  \begin{itemize}
    \item<+-> $\texttt{false} = \lambda x. \; \lambda y. \; x$
    \item<+-> $\texttt{true} = \lambda x. \; \lambda y. \; y$
  \end{itemize}

\end{frame}

\begin{frame}
  \framesubtitle{Booleans}
  \frametitle

  \[ not = ? \]

\end{frame}

\begin{frame}
  \framesubtitle{Booleans}
  \frametitle

  % we can't define the function by cases
  {

  \renewcommand\CancelColor{\color{red}}
  \[
  \xcancel{
  not(x) =
  \begin{cases}
    \texttt{true} & \mapsto \texttt{false} \\
    \texttt{false} & \mapsto \texttt{true}
  \end{cases}
  }
  \]

  }

\end{frame}

\begin{frame}
  \framesubtitle{Booleans}
  \frametitle

  \[
  \begin{aligned}
    & \texttt{false} & = \lambda x. \; \lambda y. \; x \\
    & \texttt{true} & = \lambda x. \; \lambda y. \; y
  \end{aligned}
  \]

  \[ not = \only<1>{\; ?} \only<2->{ \lambda b. \; b \; \texttt{true} \; \texttt{false} } \]

\end{frame}

\begin{frame}
  \framesubtitle{Booleans}
  \frametitle

  \[ not \; \texttt{true} \]
  \[ (\lambda b. \; b \; \texttt{true} \; \texttt{false}) \; \texttt{true} \]
  \[ (b \; \texttt{true} \; \texttt{false})[b\backslash  \texttt{true}] \]
  \[ \texttt{true} \; \texttt{true} \; \texttt{false} \]
  \[ (\lambda x. \; \lambda y. \; y) \; \texttt{true} \; \texttt{false} \]
  \[ (\lambda y. \; y)[x \backslash \texttt{true}] \; \texttt{false} \]
  \[ (\lambda y. \; y) \; \texttt{false} \]
  \[ y[y\backslash \texttt{false}] \]
  \[ \texttt{false} \]

\end{frame}

\begin{frame}
  \framesubtitle{Booleans}
  \frametitle

  \[ and = \only<1>{\; ?} \only<2->{ \lambda a. \; \lambda b. \; a \; \texttt{false} \; b } \]

\end{frame}

\begin{frame}
  \framesubtitle{Booleans}
  \frametitle

  % throughout explaining these functions, I kept saying "if a is blah then ..."
  %
  % how would we write if then else?

  \[ ifThenElse = \only<1>{\; ?} \only<2->{ \lambda b. \; \lambda x. \; \lambda y. \; b \; y \; x } \]

\end{frame}

\begin{frame}
  \framesubtitle{Pairs}
  \frametitle

  \[ \texttt{pair} = \lambda a. \; \lambda b. \; \lambda f. \; f \; a \; b \]
  \[ \texttt{fst} = \lambda p. \; p (\lambda x. \; \lambda y. \; x) \]
  \[ \texttt{snd} = \lambda p. \; p (\lambda x. \; \lambda y. \; y) \]

\end{frame}

\begin{frame}
  \framesubtitle{Natural numbers}
  \frametitle

  A natural number N is represented by N applications of a some function to some argument
  \begin{itemize}
    \item Zero - $f$ is applied 0 times - $\lambda f. \; \lambda x. \; x$
    \item One - $f$ is applied 1 times - $\lambda f. \; \lambda x. \; f \; x$
    \item Two - $f$ is applied 2 times - $\lambda f. \; \lambda x. \; f \; (f \; x)$
    \item Three - $f$ is applied 3 times - $\lambda f. \; \lambda x. \; f \; (f \; (f \; x))$
    \item \dots
  \end{itemize}

\end{frame}

\begin{frame}
  \framesubtitle{Natural numbers}
  \frametitle

  \begin{itemize}
    \item $ \texttt{zero} = \lambda f. \; \lambda x. \; x $
    \item $ \texttt{suc} = \lambda n. \; \lambda f. \; \lambda x. \; f \; (n \; f \; x) $
  \end{itemize}

\end{frame}

\begin{frame}
  \framesubtitle{Natural numbers}
  \frametitle

  \[ \texttt{suc} \; \texttt{zero} \]
  \[ (\lambda n. \; \lambda f. \; \lambda x. \; f \; (n \; f \; x)) \; \texttt{zero} \]
  \[ (\lambda f. \; \lambda x. \; f \; (n \; f \; x))[n\backslash  \texttt{zero}] \]
  \[ \lambda f. \; \lambda x. \; f \; (\texttt{zero} \; f \; x) \]

  % and if you don't believe me when I say that zero applied to f and x is the same as
  % just x, I can prove it by doing a beta reduction

\end{frame}

\begin{frame}
  \framesubtitle{Natural numbers}
  \frametitle

  \[ add = \only<1>{\; ?} \onslide<2->{ \lambda m. \; \lambda n. \; \lambda f. \; \lambda x. \; m \; f \; (n \; f \; x) } \]

  \onslide<3->{ \[ mult = \; ? \] }
  \onslide<4->{ \[ exp = \; ? \] }
  \onslide<5->{ \[ pred = \; ? \] }

\end{frame}

\section{Repetition}

\begin{frame}
  \framesubtitle{Omega}
  \frametitle

  \[ \Omega = (\lambda x. \; x \; x) \; (\lambda x. \; x \; x) \]

\end{frame}

\begin{frame}
  \framesubtitle{The Y-combinator}
  \frametitle

  \[ Y = \lambda f. \; (\lambda x. \; f \; (x \; x)) \; (\lambda x. \; f \; (x \; x)) \]

\end{frame}

\begin{frame}
  \framesubtitle{Ackermann function}
  \frametitle

  \only<1>{
  \[
  A(m, n) =
  \begin{cases}
    m = 0 & \mapsto n+1 \\
    m > 0, n = 0 & \mapsto A(m-1, 1) \\
    m > 0, n > 0 & \mapsto A(m-1, A(m, n-1)) \\
  \end{cases}
  \]
  }

  \only<2>{
  \[ pred = \dots \]

  \[ is0 = \lambda n. \; n \; (\lambda x. \; \texttt{false}) \; \texttt{true} \]
  }

  \only<3>{
  \[
  \begin{aligned}
  A & = & \\
  & Y & (\lambda f. \; \lambda m. \; \lambda n. \span \\
  & ~ & is0 &\; m  & \\
  & ~ & ~   & (is0 \; n \\
  & ~ & ~   & ~~~  (f \; (pred \; m) \; (f \; m \; (pred \; n))) \span \\
  & ~ & ~   & ~~~  (f \; (pred \; m) \; (\texttt{suc} \; \texttt{zero}))) \span \\
  & ~ & ~   & (\texttt{suc} \; n)) \span
  \end{aligned}
  \]
  }

\end{frame}

\section{Types}

\begin{frame}
  \frametitle

  Types ?
\end{frame}

\begin{frame}
  \frametitle{Applications}

  \begin{itemize}
    \item Logic (natural deduction)
    \item Category theory (cartesian closed category)
    \item Language (combinatory categorial grammars)
  \end{itemize}

\begin{frame}

\begin{frame}

\end{document}
